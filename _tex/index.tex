% Options for packages loaded elsewhere
\PassOptionsToPackage{unicode}{hyperref}
\PassOptionsToPackage{hyphens}{url}
\PassOptionsToPackage{dvipsnames,svgnames,x11names}{xcolor}
%
\documentclass[
  authoryear,
  preprint,
  1p]{elsarticle}

\usepackage{amsmath,amssymb}
\usepackage{iftex}
\ifPDFTeX
  \usepackage[T1]{fontenc}
  \usepackage[utf8]{inputenc}
  \usepackage{textcomp} % provide euro and other symbols
\else % if luatex or xetex
  \usepackage{unicode-math}
  \defaultfontfeatures{Scale=MatchLowercase}
  \defaultfontfeatures[\rmfamily]{Ligatures=TeX,Scale=1}
\fi
\usepackage{lmodern}
\ifPDFTeX\else  
    % xetex/luatex font selection
\fi
% Use upquote if available, for straight quotes in verbatim environments
\IfFileExists{upquote.sty}{\usepackage{upquote}}{}
\IfFileExists{microtype.sty}{% use microtype if available
  \usepackage[]{microtype}
  \UseMicrotypeSet[protrusion]{basicmath} % disable protrusion for tt fonts
}{}
\makeatletter
\@ifundefined{KOMAClassName}{% if non-KOMA class
  \IfFileExists{parskip.sty}{%
    \usepackage{parskip}
  }{% else
    \setlength{\parindent}{0pt}
    \setlength{\parskip}{6pt plus 2pt minus 1pt}}
}{% if KOMA class
  \KOMAoptions{parskip=half}}
\makeatother
\usepackage{xcolor}
\setlength{\emergencystretch}{3em} % prevent overfull lines
\setcounter{secnumdepth}{5}
% Make \paragraph and \subparagraph free-standing
\makeatletter
\ifx\paragraph\undefined\else
  \let\oldparagraph\paragraph
  \renewcommand{\paragraph}{
    \@ifstar
      \xxxParagraphStar
      \xxxParagraphNoStar
  }
  \newcommand{\xxxParagraphStar}[1]{\oldparagraph*{#1}\mbox{}}
  \newcommand{\xxxParagraphNoStar}[1]{\oldparagraph{#1}\mbox{}}
\fi
\ifx\subparagraph\undefined\else
  \let\oldsubparagraph\subparagraph
  \renewcommand{\subparagraph}{
    \@ifstar
      \xxxSubParagraphStar
      \xxxSubParagraphNoStar
  }
  \newcommand{\xxxSubParagraphStar}[1]{\oldsubparagraph*{#1}\mbox{}}
  \newcommand{\xxxSubParagraphNoStar}[1]{\oldsubparagraph{#1}\mbox{}}
\fi
\makeatother


\providecommand{\tightlist}{%
  \setlength{\itemsep}{0pt}\setlength{\parskip}{0pt}}\usepackage{longtable,booktabs,array}
\usepackage{calc} % for calculating minipage widths
% Correct order of tables after \paragraph or \subparagraph
\usepackage{etoolbox}
\makeatletter
\patchcmd\longtable{\par}{\if@noskipsec\mbox{}\fi\par}{}{}
\makeatother
% Allow footnotes in longtable head/foot
\IfFileExists{footnotehyper.sty}{\usepackage{footnotehyper}}{\usepackage{footnote}}
\makesavenoteenv{longtable}
\usepackage{graphicx}
\makeatletter
\newsavebox\pandoc@box
\newcommand*\pandocbounded[1]{% scales image to fit in text height/width
  \sbox\pandoc@box{#1}%
  \Gscale@div\@tempa{\textheight}{\dimexpr\ht\pandoc@box+\dp\pandoc@box\relax}%
  \Gscale@div\@tempb{\linewidth}{\wd\pandoc@box}%
  \ifdim\@tempb\p@<\@tempa\p@\let\@tempa\@tempb\fi% select the smaller of both
  \ifdim\@tempa\p@<\p@\scalebox{\@tempa}{\usebox\pandoc@box}%
  \else\usebox{\pandoc@box}%
  \fi%
}
% Set default figure placement to htbp
\def\fps@figure{htbp}
\makeatother

\makeatletter
\@ifpackageloaded{caption}{}{\usepackage{caption}}
\AtBeginDocument{%
\ifdefined\contentsname
  \renewcommand*\contentsname{Table of contents}
\else
  \newcommand\contentsname{Table of contents}
\fi
\ifdefined\listfigurename
  \renewcommand*\listfigurename{List of Figures}
\else
  \newcommand\listfigurename{List of Figures}
\fi
\ifdefined\listtablename
  \renewcommand*\listtablename{List of Tables}
\else
  \newcommand\listtablename{List of Tables}
\fi
\ifdefined\figurename
  \renewcommand*\figurename{Figure}
\else
  \newcommand\figurename{Figure}
\fi
\ifdefined\tablename
  \renewcommand*\tablename{Table}
\else
  \newcommand\tablename{Table}
\fi
}
\@ifpackageloaded{float}{}{\usepackage{float}}
\floatstyle{ruled}
\@ifundefined{c@chapter}{\newfloat{codelisting}{h}{lop}}{\newfloat{codelisting}{h}{lop}[chapter]}
\floatname{codelisting}{Listing}
\newcommand*\listoflistings{\listof{codelisting}{List of Listings}}
\makeatother
\makeatletter
\makeatother
\makeatletter
\@ifpackageloaded{caption}{}{\usepackage{caption}}
\@ifpackageloaded{subcaption}{}{\usepackage{subcaption}}
\makeatother
\journal{Psychometrika}

\usepackage[]{natbib}
\bibliographystyle{elsarticle-harv}
\usepackage{bookmark}

\IfFileExists{xurl.sty}{\usepackage{xurl}}{} % add URL line breaks if available
\urlstyle{same} % disable monospaced font for URLs
\hypersetup{
  pdftitle={Causes and effects in Dichotomous Comparative Judgments: an information-theoretical system of plausible mechanism},
  pdfauthor={Jose Manuel Rivera Espejo; Tine van van Daal; Sven De De Maeyer; Steven Gillis},
  pdfkeywords={causal inference, probability, Thurstone, comparative
judgement, directed acyclic graph, structural causal models, statistical
modeling},
  colorlinks=true,
  linkcolor={blue},
  filecolor={Maroon},
  citecolor={Blue},
  urlcolor={Blue},
  pdfcreator={LaTeX via pandoc}}


\setlength{\parindent}{6pt}
\begin{document}

\begin{frontmatter}
\title{Causes and effects in Dichotomous Comparative Judgments: an
information-theoretical system of plausible mechanism}
\author[1]{Jose Manuel Rivera Espejo%
\corref{cor1}%
}
 \ead{JoseManuel.RiveraEspejo@uantwerpen.be} 
\author[1]{Tine van Daal%
%
}
 \ead{tine.vandaal@uantwerpen.be} 
\author[1]{Sven De Maeyer%
%
}
 \ead{sven.demaeyer@uantwerpen.be} 
\author[2]{Steven Gillis%
%
}
 \ead{steven.gillis@uantwerpen.be} 

\affiliation[1]{organization={University of Antwerp, Training and
education sciences},,postcodesep={}}
\affiliation[2]{organization={University of
Antwerp, Linguistics},,postcodesep={}}

\cortext[cor1]{Corresponding author}




        
\begin{abstract}
(to do)
\end{abstract}





\begin{keyword}
    causal inference \sep probability \sep Thurstone \sep comparative
judgement \sep directed acyclic graph \sep structural causal
models \sep 
    statistical modeling
\end{keyword}
\end{frontmatter}
    

\section{Introduction}\label{sec-introduction}

Over the past decade, numerous studies have documented the effectiveness
of the \emph{comparative judgment} (CJ) method \citep{Thurstone_1927}
for assessing competencies and traits. These studies have evaluated CJ
from two main perspectives: its ability to produce reliable and valid
trait scores, and its practical applicability. Research on reliability
and validity has shown that CJ can generate precise and consistent
scores that accurately represent the traits being measured
\citep{Pollitt_2012a, Pollitt_2012b, Whitehouse_2012, vanDaal_et_al_2016, Lesterhuis_2018, Bramley_et_al_2019, Verhavert_et_al_2019, Crompvoets_et_al_2022, Bouwer_et_al_2023}.
Regarding practical applicability, several studies have highlighted CJ's
versatility across both educational and non-educational contexts,
presenting it as an efficient and effective alternative for measurement
and evaluation
\citep{Jones_2015, Bartholomew_et_al_2018, Jones_et_al_2019, Marshall_et_al_2020, Bartholomew_et_al_2020, Boonen_et_al_2020}.

Despite the growing number of CJ studies, the unsystematic and
fragmented research approaches employed in the literature have
overlooked several critical issues concerning the method. These issues
fall into three main categories: concerns about the structural model,
the measurement model, and the experimental design of CJ. In the
following sections, each issue will be discussed in detail, followed by
the introduction of an approach that addresses all three concerns
simultaneously.

A key issue in the first category is the apparent disconnect between the
method's structural and measurement models. In CJ literature, it is
common to perform data analysis and hypothesis testing on scores
estimated using the Bradley-Terry-Luce (BTL) model
\citep{Bradley_et_al_1952, Luce_1959}. Several studies use the scores
generated by the BTL model or their transformations to identify `misfit'
judges and stimuli \citep{Pollitt_2012b, vanDaal_et_al_2017}, detect
`bias' in judges' ratings \citep{Pollitt_et_al_2003, Pollitt_2012b}, or
test various hypotheses about the underlying trait being measured
\citep{Bramley_et_al_2019, Boonen_et_al_2020, Bouwer_et_al_2023, vanDaal_et_al_2017, Jones_et_al_2019, Gijsen_et_al_2021}.
However, since these scores are parameter estimates with inherent
uncertainty, the statistical literature suggests that separating the
analysis from this uncertainty may artificially inflate the precision
and power of the results. This, in turn, could increase the risk of
committing a type I error, where a null hypothesis is wrongly rejected
\citep{McElreath_2020}. To address this issue properly, the approach
should follow a strategy similar to that used in Structural Equation
Modeling (SEM), where data analysis and hypothesis testing occur at the
structural model level, while the BTL model functions as the measurement
model.

\section{Theory}\label{sec-theory}

\subsection{Let's talk about Thurstone co.}\label{sec-theory-thurstone}

\subsection{A scientific model for the CJ}\label{sec-theory-scientific}

\subsection{From theory to statistical
model}\label{sec-theory-statmodel}

\section{Discussion}\label{sec-discuss}

\subsection{Findings}\label{sec-discuss-finding}

\subsection{Limitations and further
research}\label{sec-discuss-limitations}

\section{Conclusion}\label{sec-conclusion}

\newpage{}

\section*{Declarations}\label{declarations}
\addcontentsline{toc}{section}{Declarations}

\textbf{Funding:} The project was founded through the Research Fund of
the University of Antwerp (BOF).

\textbf{Financial interests:} The authors have no relevant financial
interest to disclose.

\textbf{Non-financial interests:} Author XX serve on advisory broad of
Company Y but receives no compensation this role.

\textbf{Ethics approval:} The University of Antwerp Research Ethics
Committee has confirmed that no ethical approval is required.

\textbf{Consent to participate:} Not applicable

\textbf{Consent for publication:} All authors have read and agreed to
the published version of the manuscript.

\textbf{Availability of data and materials:} No data was utilized in
this study.

\textbf{Code availability:} All the code utilized in this research is
available in the digital document located at:
\url{https://jriveraespejo.github.io/paper2_manuscript/}.

\textbf{Authors' contributions:} \emph{Conceptualization:} S.G., S.DM.,
T.vD., and J.M.R.E; \emph{Methodology:} S.DM., T.vD., and J.M.R.E;
\emph{Software:} J.M.R.E.; \emph{Validation:} J.M.R.E.; \emph{Formal
Analysis:} J.M.R.E.; \emph{Investigation:} J.M.R.E; \emph{Resources:}
S.G., S.DM., and T.vD.; \emph{Data curation:} J.M.R.E.; \emph{Writing -
original draft:} J.M.R.E.; \emph{Writing - review \& editing:} S.G.,
S.DM., and T.vD.; \emph{Visualization:} J.M.R.E.; \emph{Supervision:}
S.G. and S.DM.; \emph{Project administration:} S.G. and S.DM.;
\emph{Funding acquisition:} S.G. and S.DM.

\newpage{}

\section{Appendix}\label{sec-appendix}

\newpage{}

\subsection*{References}\label{references}
\addcontentsline{toc}{subsection}{References}

\renewcommand{\bibsection}{}
\bibliography{references.bib}





\end{document}
